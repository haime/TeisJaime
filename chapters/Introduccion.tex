\chapter{Introducción}

\section{Contexto}
Mas allá de la compleja circuiteria y mecánica que conforman la parte física, los robots móviles rigen su comportamiento en base a el software, dependiendo de cuanta interacción tenga el robot con su entorno, podría llegar a ser mas complejo que todo el hardware que lo conforma. Una parte importante de este software, es la forma en la que el robot puede obtener datos e interpretarlos. 

El sistema de visión humano, es al que más recurre para obtener información de su entorno. Por ello no es de extrañarse que la visión computacional, en la robótica, tenga una participación muy importante,por que estas maquinas empiezan a ser utilizadas, para tareas que antes solo los humanos realizaban. Entonces las deben realizar de una manera adecuada y en tiempo. 

El tiempo es preciado en cualquier rama y esta no es la excepción, la forma de ganar tiempo que se a venido sesgando por hardware es el paralelismo, y no solo hablo de procesadores multinúcleo, las tarjetas gráficas se pueden programar para realizar tareas de propósito general.   

\section{Problema a resolver}
Los algoritmos que se utilizan, en la visión computacional muchas veces son muy confiables, pero consumen mucho tiempo de procesador, por esto se ha tratado de hacer mas eficientes estos algoritmos, pero provoca que  la confiabilidad de estos disminuya. El tiempo en el cual se adquieren y procesa la información, es crucial en la actividad de un robot, de esto depende que decisión tomara.

Con lo anteriormente dicho, lo importante es el tiempo en que procesemos los datos, para tomar una decisión, pero igual de importante es que la información obtenida sea congruente. 

En muchos casos, el software que funciona en paralelo es mas rápido que el secuencial, podemos ver que los algoritmos que se manejan en visión computacional son siempre secuenciales. Otro punto importante son los recursos, como procesador y memoria de la computadora del robot, siempre estarán siendo demandados por otros módulos del robot.    

\section{Hipótesis} 



\subsection{Alcance}

\section{Estructura de la tesis}
