\chapter{Trabajo a Futuro}

Como se vio en el capitulo anterior hay nuevas necesidades para este Robot Justina, y con un aceleramiento de 5 veces no me parece suficiente  por lo que hay que seguir trabajando en esto. Hay que hacer pruebas diferentes tarjetas gráficas más poderosas y de nueva generación.

Pero no solo el hardware es lo que hay que probar, como vimos en el capitulo tres, los diferentes tipos de memoria ayudan a hacer más eficiente el acceso a los datos, experimentar con estos tipos de memoria para buscar un mejor desempeño. Otro aspecto que se podría mejorar es al momento de lanzar el Kernel ya que hay que buscar cual es la configuración adecuada para el lanzamiento, con esto me refiero al numero de bloques e hilos que ejecutaran el programa. 

Encontrar otra forma más rápida de pasar los datos de la memoria RAM a la memoria de la GPU ya que ese es un cuello de botella importante, encontrar otra forma de manejar las imágenes para que los accesos a memoria sena más rápidos. En fin tratar de sacarle todo el jugo a la tarjeta grafica adaptando el código a esta.  

Algo que ayudaría a Justina, es desarrollar el algoritmo para que genere el descriptor de SIFT y otro para hacer el emparejamiento de estos descriptores en paralelo, no se si usar las tarjetas gráficas para hacer el emparejamiento sea lo optimo pero igual se puede intentar y ver si arroja resultados favorables.  