\chapter{Conclusiones y Trabajo a Futuro}
\spacing{1.5}
\section{Conclusiones}
Las actividades que el robot de servicio Justina desempeña son más rigurosas cada año, un ejemplo de esto es una nueva prueba propuesta en Robocup 2015 llamada \textit{Manipulation and object recognition}, donde se tiene que buscar objetos en un librero. Para esta prueba solo se cuenta con 3 minutos para reconocer y manipular objetos,  de los cuales casi todo el tiempo se invierte en reconocimiento, ya que con una sola foto no podríamos analizar todos los objetos que  hay en el librero, por lo cual, hay que hacer varias tomas de diferente ángulos. Para el reconocimiento de objetos el equipo de Biorobotica de la UNAM usa la siguiente estrategia: se segmentan los objetos y se procesa solo una cantidad pequeña de pixels para el reconocimiento, usando un sensor Kinect para obtener las imágenes, este tiene una resolución de  640 x 480 px, y se utiliza el código de openSIFT para procesar estas imágenes. Desafortunadamente el intento por segmentar los objetos de un librero no han sido existo, por lo que hay que analizar imágenes más grandes y desde una mayor distancia. La implicación de esto es la necesidad de mayor resolución en las imágenes para no perder detalle de los objetos y tener una manera más rápida de procesar estas nuevas imágenes. Por esto se adquirió una cámara RGB Logitech C920 que tiene una resolución de 1920 x 1080 px, y el procesamiento de la imagen fue mucho más lento pero mostro una mejora en cuanto a reconocimiento. Se planea empezar a usar también el nuevo sensor de Microsoft Kinect 2 el cual también tiene una resolución de 1920 x 1080 px.\\
El objetivo de este trabajo fue obtener una manera de procesar más rápido estas imágenes de alta resolución. Y podemos decir que el objetivo se cumplió, ya que obtuvimos una mejora en el desempeño, como se puede ver en el capítulo Pruebas y Resultados, ya que en promedio se obtuvo un speedup de 4.94, es un buen resultado en general. Aunque analizando los resultados encontraremos los mejores cuando la resolución de la imagen es de 1280 x 960 px, esto lo podemos relacionar con la ocupación de los SM en la tabla 5-1 encontramos un máximo y un mínimo, cuando una imagen es grande la ocupación estará en su máximo pero si la imagen es pequeña la ocupación no será la más óptima. Entonces, la manera más óptima de utilizar esta implementación de SIFT, en GPU, será utilizando imágenes de alta resolución para obtener los ejores tiempos sin perder detalles por no tener suficiente información.      
\section{Trabajo a Futuro}
Como se vio, hay nuevas necesidades para el robot Justina, por lo que hay que seguir trabajando en esto. Se deben hacer pruebas en diferentes tarjetas gráficas más poderosas y de nueva generación. Pero no solo el hardware es lo que hay que probar, como vimos en el capítulo tres, los diferentes tipos de memoria ayudan a hacer más eficiente el acceso a los datos, experimentar con estos tipos de memoria para buscar un mejor desempeño.\\
Encontrar otra forma más rápida de pasar los datos de la memoria RAM a la memoria de la GPU (ese es un cuello de botella importante) y encontrar otra forma de manejar las imágenes para que los accesos a memoria sean más rápidos.\\
Algo que ayudaría a Justina, es desarrollar el algoritmo para que genere el descriptor de SIFT y otro para hacer el emparejamiento de estos descriptores en paralelo. Este último podría ser el más importante ya que una vez que se obtienen los descriptores, compararlos con todos los encontrados en una imagen es una tarea muy sencilla, pero ardua ya que la cantidad de descriptores que se pueden encontrar en una imagen de alta resolución es grande, y la cantidad de descriptores del objeto también es alta, haciendo el tiempo de esta tarea muy alto. \nocite{AIShack} \nocite{dicc}

 
 
