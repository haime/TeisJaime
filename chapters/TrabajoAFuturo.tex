 
 
 \chapter{Conclusiones y Trabajo a Futuro}

\section{Conclusiones}

 Las pruebas a las que es sometido el robot de servicio Justina son más rigurosas, por este motivo los sistemas deben ser cada vez más rápidos y confiables. También las nuevas generaciones de sensores son mucho mejores tenemos más información para analizar, actualmente se utiliza un sensor Kinect para obtener las imágenes, esta tiene una resolución de  640 x 480 px y se utiliza el código de openSIFT para procesar esta imágenes, después     se adquirió una cámara RGB Logitech C920 que tiene una  de 1920 x 1080 px, el procesamiento de la imagen fue mucho más lento. Se planea empezar a usar también el nuevo sensor de Microsoft Kinect 2 el cual también tiene una resolución de 1920 x 1080 px. 
 
 
 Los resultados obtenidos tiene dos matices en mi opinion, donde las imágenes son muy pequeñas y hay poco que procesar , cuando el robot necesita encontrar objetos en una mesa, segmenta los objetos de la mesa y quedan solo los objetos, que no tienen una cantidad muy grande de pixeles a procesar, por lo que aun que es un poco más rápido procesarlos en el GPU, no tiene gran aceleración en comparación a la librería de OpenCV y es mucho más sencillo implementarlo; la otra parte es que se ha estado trabajando para que Justina no solo encuentre objetos en la mesa sino que debe tener capacidad para encontrarlos en un libreo o estante. En el laboratorio de BioRobotica de la UNAM  se ha estado tratando de segmentar estos estantes o libreros para poder seguir con la misma metodología al momento de buscar objetos, cosa que no se ha logrado con éxito total. Lo que se implemento fue hacer una búsqueda de los objetos en toda la imagen, para esto se intentó hacer con las imágenes que entrega el Kinect, pero la baja calidad de la imagen no permitía encontrar los objetos, ya que la distancia de donde se tomaba la foto era más lejos de lo que acostumbra ser.
 Se cambió la fuente de la imagen por la cámara C920 y los resultados mejoraron, pero el tiempo de procesamiento aumento.

 EL robot Justina participa en una competencia principalmente llamada Robocup, en 2015 se puso una nueva prueba llamada  \textit{Manipulation and object recognition} donde se tenía que buscar objetos en un librero y manipularlos básicamente. Para esta prueba solo dan 3 minutos para reconocer y manipular objetos,  de los cuales nosotros gastamos todo el tiempo solo reconociendo, ya que con una sola foto no podríamos analizar todos los objetos que  hay en el librero, por lo cual hay que hacer varias tomas de diferente ángulos.


 
 El punto a resaltar es que la cantidad de datos que se tienen que procesar crecieron solo tres veces y se tarda en procesar hasta 13 veces más tiempo que con las anteriores resoluciones, lo que empieza a sonar como una buena idea es voltear a ver a las tarjetas gráficas de propósito general, las cuales ya se cuentan con ellas en las laptops que el robot usa. Observando los resultados con resoluciones de 1920 x 1080 podemos ver una aceleración de hasta 5 veces respecto a OpenSIFT y 2 veces contra la librería OpenCV, ya que en lugar de que el proceso tarde 5 segundos por foto estaría tardando solo un segundo.  
 

\section{Trabajo a Futuro}

 Como se vio hay nuevas necesidades para este Robot Justina, y con un aceleramiento de 5 veces no me parece suficiente  por lo que hay que seguir trabajando en esto. Hay que hacer pruebas en diferentes tarjetas gráficas más poderosas y de nueva generación.

 Pero no solo el hardware es lo que hay que probar, como vimos en el capítulo tres, los diferentes tipos de memoria ayudan a hacer más eficiente el acceso a los datos, experimentar con estos tipos de memoria para buscar un mejor desempeño. Otro aspecto que se podría mejorar es al momento de lanzar el Kernel ya que hay que buscar cual es la configuración adecuada para el lanzamiento, con esto me refiero al número de bloques e hilos que ejecutaran el programa. 

 Encontrar otra forma más rápida de pasar los datos de la memoria RAM a la memoria de la GPU ya que ese es un cuello de botella importante, encontrar otra forma de manejar las imágenes para que los accesos a memoria será más rápidos. En fin tratar de sacarle todo el jugo a la tarjeta gráfica adaptando el código a esta.  

 Algo que ayudaría a Justina, es desarrollar el algoritmo para que genere el descriptor de SIFT y otro para hacer el emparejamiento de estos descriptores en paralelo, no sé si usar las tarjetas gráficas para hacer el emparejamiento sea lo óptimo pero igual se puede intentar y ver si arroja resultados favorables.  
 
 
