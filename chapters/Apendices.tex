\chapter{Kernel Convolución}
\begin{small}



\lstset{language=C,
                basicstyle=\ttfamily,
                keywordstyle=\color{blue}\ttfamily,
                stringstyle=\color{red}\ttfamily,
                commentstyle=\color{green}\ttfamily,
        frame= single,
        numbers = none,
        xleftmargin=0.5em,
        framexleftmargin=0.5em        
}
\begin{lstlisting}
__global__ void Convolution(float* image,float* mask, 
		ArrayImage* PyDoG, int maskR,int maskC,
	        int imgR,int imgC, float* imgOut, int idxPyDoG)
{
int tid= threadIdx.x;
int bid= blockIdx.x;
int bDim=blockDim.x;
int gDim=gridDim.x;
int iImg=0;
float aux=0;
int pxlThrd = ceil((double)(imgC*imgR)/(gDim*bDim)); 
for(int i = 0; i <pxlThrd; ++i)
	{
		
iImg=(tid+(bDim*bid)) + (i*gDim*bDim); 
 if(iImg < imgC*imgR){
  int condition=maskC/2+imgC*(floor((double)maskC/2));
  if (iImg-condition < 0  ||										
	 iImg+condition > imgC*imgR ||								
	 iImg%imgC < maskC/2 ||										
	 iImg%imgC > (imgC-1)-(maskC/2) )							
	 {
	 	aux=0;
	 }else{		
		int itMask = 0;
		int itImg=iImg-condition;
		for (int j = 0; j < maskR; ++j)
		{		
			for (int h = 0; h < maskC; ++h)
			{
				aux+=image[itImg]*mask[itMask];
				++itMask;
				++itImg;
			}
			itImg+=imgC-maskC;
		}
	 }
	imgOut[iImg]=aux;
	aux=0;
  }
 }
 PyDoG[idxPyDoG].image=imgOut;
}
\end{lstlisting}

\end{small}
\pagebreak

\chapter{Kernel Localización de máximos y mínimos }

\begin{small}
\lstset{language=C,
                basicstyle=\ttfamily,
                keywordstyle=\color{blue}\ttfamily,
                stringstyle=\color{red}\ttfamily,
                commentstyle=\color{green}\ttfamily,
        frame= single,
        numbers = none,
        xleftmargin=0.5em,
        framexleftmargin=0.5em        
}
\begin{lstlisting}
__global__ void LocateMaxMin(ArrayImage* PyDoG, int idxPyDoG ,
		float * imgOut ,MinMax * mM, int maskC, int imgR,
		int imgC, int idxmM)
{
int tid= threadIdx.x;
int bid= blockIdx.x;
int bDim=blockDim.x;
int gDim=gridDim.x;
int iImg=0;
int pxlThrd = ceil((double)(imgC*imgR)/(gDim*bDim)); 
for(int i = 0; i <pxlThrd; ++i)
 {
  int min=0;
  int max=0;
  float value=0.0;
  float compare =0.0;
  iImg=(tid+(bDim*bid)) + (i*gDim*bDim); 
  if(iImg < imgC*imgR){
	int condition=maskC/2+imgC*(floor((double)maskC/2));
	if (iImg-condition < 0  ||										
		iImg+condition > imgC*imgR ||								
		iImg%imgC < maskC/2 ||										
		iImg%imgC > (imgC-1)-(maskC/2) )							
	{                  
		imgOut[iImg]=0;				
	}
	else{
		value=PyDoG[idxPyDoG].image[iImg];
		for (int m = -1; m < 2; ++m)
		{
		  int itImg=iImg-(1+imgC);
		  for (int j = 0; j < 3; ++j)
		  {		
		    for (int h = 0; h < 3; ++h)
			{
				compare =PyDoG[idxPyDoG+m].image[itImg];
				if(value<=compare && max==0)
				{
					++min;
				}
				else if(value>=compare && min==0)
				{
					++max;
				}
				++itImg;
			  }
			  itImg+=imgC-3;
		  }
		}
 		if( (min==26 || max==26)) {
			imgOut[iImg]=1;
		}else{
			imgOut[iImg]=0;
		}
	}
  }
 }
mM[idxmM].minMax=imgOut;
}
\end{lstlisting}

\end{small}
\pagebreak
\chapter{Kernel Remover puntos malos}

\begin{small}
\lstset{language=C,
                basicstyle=\ttfamily,
                keywordstyle=\color{blue}\ttfamily,
                stringstyle=\color{red}\ttfamily,
                commentstyle=\color{green}\ttfamily,
        frame= single,
        numbers = none,
        xleftmargin=0.5em,
        framexleftmargin=0.5em        
}
\begin{lstlisting}
__global__ void RemoveOutlier(ArrayImage* PyDoG, MinMax * mM,
	int idxmM, int idxPyDoG, int imgR,int imgC ,float* auxOut)
{
int tid= threadIdx.x;
int bid= blockIdx.x;
int bDim=blockDim.x;
int gDim=gridDim.x;
int iImg=0;
int pxlThrd = ceil((double)(imgC*imgR)/(gDim*bDim)); 
for(int i = 0; i <pxlThrd; ++i 
{

  iImg=(tid+(bDim*bid)) + (i*gDim*bDim); 
		
  if(iImg < imgC*imgR){
	if(mM[idxmM].minMax[iImg]>0 && PyDoG[idxPyDoG].image[iImg]>0.05)
	{
	  float d, dxx, dyy, dxy, tr, det;
	  d = PyDoG[idxPyDoG].image[iImg];
	  dxx = PyDoG[idxPyDoG].image[iImg+1]+
	  	PyDoG[idxPyDoG].image[iImg-1] - (2*d);
	  dyy = PyDoG[idxPyDoG].image[iImg+imgC]+
	  	PyDoG[idxPyDoG].image[iImg-imgC] - (2*d);
	  dxy = (PyDoG[idxPyDoG].image[iImg+1+imgC]-
	  	PyDoG[idxPyDoG].image[iImg-1+imgC] - 
	  	PyDoG[idxPyDoG].image[iImg+1-imgC] +
	  	PyDoG[idxPyDoG].image[iImg-1-imgC])/4.0;
	  tr = dxx + dyy;
	  det = dxx*dyy - dxy*dxy;
		
	  if(det<=0 && !(tr*tr/det < 12.1))
	  mM[idxmM].minMax[iImg]=0;
	 }else
	 {
	   mM[idxmM].minMax[iImg]=0;
	 }
     auxOut[iImg]=mM[idxmM].minMax[iImg];
	}
}
	
}
\end{lstlisting}

\end{small}
\pagebreak

\chapter{Kernel Asignar magnitud y orientación}

\begin{small}
\lstset{language=C,
                basicstyle=\ttfamily,
                keywordstyle=\color{blue}\ttfamily,
                stringstyle=\color{red}\ttfamily,
                commentstyle=\color{green}\ttfamily,
        frame= single,
        numbers = none,
        xleftmargin=0.5em,
        framexleftmargin=0.5em        
}
\begin{lstlisting}
__global__ void OriMag(ArrayImage* PyDoG, int idxPyDoG,
	int imgR,int imgC , ArrayImage* Mag, ArrayImage* Ori,
	int idxMagOri, float* MagAux, float* OriAux) 
{
int tid= threadIdx.x;
int bid= blockIdx.x;
int bDim=blockDim.x;
int gDim=gridDim.x;
float dx,dy;
int iImg=0;
int pxlThrd = ceil((double)(imgC*imgR)/(gDim*bDim)); 
for(int i = 0; i <pxlThrd; ++i)
{
	iImg=(tid+(bDim*bid)) + (i*gDim*bDim); 
	if(iImg < imgC*imgR){
	int condition=1/2+imgC*(floor((double)1/2));
	if (iImg-condition < 0  ||										
	iImg+condition > imgC*imgR ||								
	iImg%imgC < 1/2 ||										
	iImg%imgC > (imgC-1)-(1/2) )							
	{                  
		OriAux[iImg]=0;
		MagAux[iImg]=0;
	}
	else{
		dx=PyDoG[idxPyDoG].image[iImg+1]-
			PyDoG[idxPyDoG].image[iImg-1];
		dy=PyDoG[idxPyDoG].image[iImg+imgC]-
			PyDoG[idxPyDoG].image[iImg-imgC];
		MagAux[iImg]=sqrt(dx*dx + dy*dy);
		OriAux[iImg]=atan2(dy,dx);
    }
}
}
	
	Mag[idxMagOri].image= MagAux;
	Ori[idxMagOri].image= OriAux;

	
}
\end{lstlisting}


\end{small}
\pagebreak
\chapter{Kernel Generar puntos característicos}
\begin{small}

	\lstset{language=C,
                basicstyle=\ttfamily,
                keywordstyle=\color{blue}\ttfamily,
                stringstyle=\color{red}\ttfamily,
                commentstyle=\color{green}\ttfamily,
        frame= single,
        numbers = none,
        xleftmargin=0.5em,
        framexleftmargin=0.5em        
}
\begin{lstlisting}
__global__ void KeyPoints(ArrayImage * Mag,
	ArrayImage * Ori, MinMax * mM , int idxMOmM,
	keyPoint * KP, float sigma, int imgR,int imgC,
	int octava )
{
int tid= threadIdx.x;
int bid= blockIdx.x;
int bDim=blockDim.x;
int gDim=gridDim.x;
float o = 0;
int x=0, y=0, octv=-1;
int iImg=0;
int pxlThrd = ceil((double)(imgC*imgR)/(gDim*bDim)); 
for(int i = 0; i <pxlThrd; ++i) 
{
	iImg=(tid+(bDim*bid)) + (i*gDim*bDim);  
	octv=-1;
	if(iImg < imgC*imgR ){
	  if(mM[idxMOmM].minMax[iImg]>0){
		int condition=9/2+imgC*(floor((double)9/2));
		if (iImg-condition < 0  ||														iImg+condition > imgC*imgR ||												iImg%imgC < 9/2 ||										
			iImg%imgC > (imgC-1)-(9/2) )							
		{                  
		  o=-1.0;
		  x=-1;
		  y=-1;
		  octv=-1;
		}
		else{
		  float histo[36]={0,0,0,0,0,0,0,0,0,0,0,0,0,0,0,0,0,0,0,
		  	0,0,0,0,0,0,0,0,0,0,0,0,0,0,0,0,0};
		  octv=octava;
		  x=iImg%imgC;
		  y=iImg/imgC;
		  int idxMO= (iImg-4)-(4*imgC);
		  float exp_denom = 2.0 * sigma * sigma;
		  float w;
		  int bin;
		  for (int i = -4; i < 5; ++i)
		  {
		    for (int j = -4; j < 5; ++j)
			{
			  w = exp( -( i*i + j*j ) / exp_denom );
			  bin=round((double)(36*Ori[idxMOmM].image[idxMO])
			  			/6.28318530718);
	  		  bin = ( bin < 36 )? bin : 0;
	  		  histo[bin]= w*Mag[idxMOmM].image[idxMO];
	  		  ++idxMO;
			 }
			 idxMO=idxMO+imgC-9;
		   }
		   int idxH=0;
		   float valMaxH = histo[0];
		   for (int i = 1; i < 36; ++i)
		   {
		    if(histo[i]>valMaxH){
		      idxH = i;
		    }
		   }
		   int l = (idxH == 0)? 35:idxH-1;
		   int r = (idxH+1)%36;
		   float bin_= bin + ((0.5*(histo[l]-histo[r]))/
		   		(histo[l]-(2*histo[idxH])+histo[r]));
		   bin_= ( bin_ < 0 )? 36 + bin_ :
		   ( bin_ >= 36 )? bin_ - 36 :
		   bin_;
		   o=((6.28318530718*bin_)/36)-3.141592654;
	    }
	  }
	  KP[iImg].orientacion=o;
	  KP[iImg].x=x;
	  KP[iImg].y=y;
	  KP[iImg].octv=octv;
	}
}
}
\end{lstlisting}

\end{small}
\pagebreak
\chapter{Kernel Generar descriptor}

